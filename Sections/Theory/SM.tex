\section{The Standard Model}
\label{sec:SM}
The current best description we have of our universe today is the Standard Model. It includes the elementary particles and the forces acting between them and can explain all visible matter we have around us (sans gravity). 

The particles we have in the SM are organized in three generations, as shown in figure \ref{fig:SM}. It consists of two different types of particles, namely fermions and bosons. Fermions are the matter particles, and the bosons carry the forces that act between the fermions and are usually called force particles. They are usually arranged in doublets according to their isospin, that describes their electroweak coupling. 

\begin{figure}[H]
    \centering
    \includegraphics[width=\textwidth]{Figures/FromOnline/SM.png}
    \caption{An overview of the particles in the Standard Model \cite{SMpicture}.}
    \label{fig:SM}
\end{figure}

\subsection{Fermions}
If we look at figure \ref{fig:SM}, we can see that there are 12 different fermions split up in two groups called quarks and leptons, which again are split up into three generations. The lightest quarks and leptons are in the first generation, and the most massive quarks are in the third generation. All of the fermions are $\frac{1}{2}$-spin particles and differ by mass and charge. They also obey the Pauli exclusion principle, which means that only one fermion can occupy a given quantum state for any given quantum number.

\subsubsection{Quarks}
We have six different quarks \cite{thomson} in the SM, and they differ by mainly mass, but also charge (color charge, as opposed to electrical charge). All the up-type quarks, which are the three quarks in the first row in figure \ref{fig:SM}, have charge +2/3 and all the down-type quarks, which is in the second row in figure \ref{fig:SM}, have charge -1/3. Since they carry a non-zero electric charge, they interact via the quantum electrodynamics (QED). They can also interact with the weak force which allows all possible combinations in doublets, where they differ by one unit of charge. 

\begin{align}
    \binom{u}{d} \text{,  } \binom{u}{s} \text{,  } \binom{u}{b} \text{,  } \binom{c}{d} \text{,  } \binom{c}{s} \text{,  } \binom{c}{b} \text{,  } \binom{t}{d} \text{,  } \binom{t}{s} \text{,  } \binom{t}{b}
\end{align}

The two quarks in the first generation, namely up and down, are the particles can combine to form what are commonly called protons and neutrons - which then combine in various ways to form atomic nuclei. The second and third generation of quarks (charm, strange, top and bottom) are more exotic and more massive than the quarks in the first generation (up and down). Since the quarks in the second and third generations are more massive, they need more energy to exist and will therefore decay to lighter particles very fast. Quarks are the only known particles that interact with all the fundamental forces. They also carry color charge in addition to electrical and weak charge, which allows them to couple to different gluons through the strong interaction in quantum chromodynamics (QCD). 

Note that the quarks can be described by either mass eigenstates corresponding to the free-propagating Hamiltonian, or weak eigenstates corresponding to the way they interact under the weak interaction. Those sets of eigenstates do not coincide but are instead related to each other by the CKM matrix \cite{thomson}. 




\subsubsection{Leptons}
Leptons \cite{thomson} are the other six fermions in the SM. Like quarks, they come in three generations and are arranged in doublets, as shown below.

\begin{align}
    \binom{\nu_e}{e^-} \text{,  } \binom{\nu_\mu}{\mu} \text{,  } \binom{\nu_\tau}{\tau}
\end{align}

They also differ by mass and charge (electrical) where the lower component has charge -1, and the upper has no charge. Because of that, neutrinos only interact with weak bosons, while for the electron, muon and, tau, electromagnetic interactions are also possible. The neutrinos are also almost massless, unlike the other fermions, and were initially assumed to be massless. 

Analogously to quarks, fermion mass and weak eigenstates do not coincide and they are related by the PMNS matrix \cite{thomson}. This phenomenon provides an explanation of neutrino oscillations, where after some distance from the interaction point we measure a different neutrino flavor. Recent measurements of neutrino oscillations indicate that neutrinos are massive - although the SM cannot explain this. 


\subsection{Bosons}
On the right-hand side of figure \ref{fig:SM}, the integer spin particles (spin-1) known as bosons \cite{thomson} are shown. Bosons follow Bose-Einstein statistics and contain the force carriers for the electromagnetic force, the weak force, and the strong force. If gravity was included in the Standard Model, it would have been through an extra boson, the graviton ($G$).

\subsubsection{The strong nuclear force}
The strong nuclear force is mediated by the gluon (g) \cite{thomson} and couples to the three color charges r, g, b. Moreover, as suggested by the name, it has a very strong coupling to quarks, and it is responsible for the fact that quarks do not appear as free, unbounded particles. The gluon is massless, and unlike the other forces, it can couple with itself, as you can see in the Feynman diagrams\footnote{Feynman diagrams are a way to graphically represent interactions between particles.} below.

\begin{figure}[H]
    \centering
    \includegraphics[width = \textwidth]{Figures/FeynmanDiagrams/gluon.pdf}
    \caption{Feynman diagrams fof the strong force vertices, where the curly lines represent the gluons \cite{STRONGforce}.}
    \label{fig:my_label}
\end{figure}

\subsubsection{The electromagnetic force}
The electromagnetic force is mediated by the photon ($\gamma$) \cite{thomson} and couples to all fermions with an electrical charge, namely all the quarks and leptons except for the neutrinos. As for the gluon, it has no mass, but it cannot couple to itself. 

\begin{figure}[H]
    \centering
    \includegraphics[width = 0.4\textwidth]{Figures/FeynmanDiagrams/photon.png}
    \caption{A Feynman diagram of the electromagnetic vertex, where the wiggly line represent the photon \cite{EMforce}.}
    \label{fig:my_label}
\end{figure}

\subsubsection{The weak nuclear forces}
The weak nuclear forces are mediated by the $W^\pm$-bosons and the $Z^0$-boson \cite{thomson}. They couple to all particles in the SM, and unlike the other forces, they also have a mass. This feature is explained in the GWS model \cite{Xin2007GlashowWeinbergSalamMA} through a spontaneous symmetry breaking due to a non-zero vacuum expectation value that adds an extra degree of freedom to otherwise massless force carriers. It is also the only interaction known to allow flavor change, which means that e.g., an up quark can become a down quark or a muon become an electron. 

\begin{figure}[H]
    \centering
    \includegraphics[width = 0.8\textwidth]{Figures/FeynmanDiagrams/weak.png}
    \caption{Feynman diagrams of the weak force vertices \cite{WEAKforce}.}
    \label{fig:my_label}
\end{figure}


\subsubsection{SM symmetries and the Higgs boson}
Formally, the three interactions in the nature are related by to three different symmetry groups. The SM in fact is represented as

\begin{equation}
    U(1)_Y \otimes SU(2)_L \otimes SU(3)_C,
\end{equation}

where $U(1)_Y$ is the symmetry associated to the hyper charge $Y$, $SU(2)_L$ the transformations acting on the left-handed isospin doublets and $SU(3)_C$ are the ones acting on color triplets. The physical bosons $\gamma, Z^0, W^\pm$ are not the mediator associated to the single symmetry groups, but are instead a linear combination of those states explained by the spontaneous symmetry breaking leading to the electroweak theory. 

The final piece, the Higgs boson (H) \cite{thomson}, that was missing in the SM was discovered in 2012 at CERN \cite{Higgs_ATLAS, Higgs_CMS}. The discovery gives increased credence to the SM and explains why the weak force carriers, $Z^0$ and $ W^\pm $, have mass. The fermion masses, except for neutrinos, can also be explained by couplings to the Higgs boson. 

