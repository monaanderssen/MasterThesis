\chapter{Machine learning or cut and count?}
\label{sec:results}
Until this point we have looked at how well the different analyses perform on the same dataset and we are now going to see how sensitive they are to the signal samples we have chosen. To be able to do this we have to know how we can quantify the sensitivity before we move on to compare ML to cut and count analysis in searching for Supersymmetry and Dark Matter. 
%\input{Sections/Results/CutAndCount/SlepSlep}
%%\section{Cut and count}
%\label{sec:resultsCandC}

\input{Sections/Results/CutAndCount/SlepSlep}

\input{Sections/Results/CutAndCount/C1C1_SlepSnu}

\input{Sections/Results/CutAndCount/C1C1_WW}

\input{Sections/Results/CutAndCount/Mono-Z}

%\section{Machine learning}
\label{sec:resultsML}

\subsection{BDT}
\label{sec:resultsBDT}

\subsubsection{MC simulated data}

\subsubsection{Real data}




\subsection{Neural networks}
\label{sec:resultsNN}

\subsubsection{MC simulated data}

\subsubsection{Real data}

\section{Quantifying the sensitivity}
\label{sec:significance}

The overall goal of the analysis carried out in this thesis is to optimize the sensitivity to discover new physics and possibly in the end to claim discovery of new particles. To check if we have succeeded in making our model sensitive to new physics we can check the expected p-values and significances for the signal regions considered. 

Each event in the data will either pass or fail the cuts defining the signal region used in the analysis. The number of events therefore follows a binomial distribution, and the probability of finding $n$ events in the signal region is given by equation \ref{eq:pval1}. 

\begin{equation}
    \label{eq:pval1}
    P(n|N, p) = \frac{N!}{n!(N-n)!} p^n(1-p)^{N-n},
\end{equation}

where $N$ is the total number of events and p the probability for the event to pass the cuts. If the number of total events gets very big and the probability $p$ gets very small, we can approximate equation \ref{eq:pval1} as a Poisson distribution, which is given by

\begin{equation}
    \label{eq:pval2}
    P(n|\nu) = \frac{\nu^n}{n!} e^{-\nu},
\end{equation}

where $\nu$ is given by a hypothesis. This equation (eq. \ref{eq:pval2}) gives us a probability of $n$ observed events when the expectation is to observe $\nu$ events. 

When we talk about hypotheses in this thesis, we mean \textit{background-only} (b-only) and \textit{signal + background} (s+b) hypotheses. The b-only hypothesis is the SM, which is a well known theory. We are going to check this against the s+b hypothesis which in our case will be one of the three SUSY processes or the mono-Z process. This gives us the opportunity to introduce the p-values. A p-value is a measure to judge the deviation of the observed number of events from the b-only hypothesis expectations. If we assume that only the SM processes contribute we can use the p-value to get the probability of observing as many, or more, events as we can find in the data. With a perfectly known background the p-value can be found by

\begin{equation}
    \label{eq:pvalues}
    p = \sum_{n=n_{obs}}^\infty f(n;b),
\end{equation}


where $f(n;b)$ is the Poisson distribution given in equation \ref{eq:pval2}. It is common to convert the p-value into an observed significance $z$ with a unit Gaussian $\Phi$. This is given by

\begin{equation}
    \label{eq:signifobs}
    z = \Phi^{-1}(1-p)
\end{equation}

 and the unit is expressed with a $\sigma$ as in number of standard deviations from the center of the Gaussian distribution. To claim discovery of a new particle we need an observed significance above 5 $\sigma$. We are not going to check the observed significance in this thesis, but calculate the expected significance to see if there is any hope to discover new physics for the models we are looking at. The expected significance is the significance associated with the median number of expected events under the s+b hypothesis. It quantifies how well the two hypotheses we are looking at are separated. This is given by equation \ref{eq:signifobs}, where the p-value now is expressed as a \textit{confidence level} (CL) given as
 
 \begin{equation}
    \label{eq:CL}
    CL_{s+b} = \sum_{n=0}^{q_{obs}} \frac{(s+b)^n}{n!} e^{-(s+b)},
 \end{equation}

where $q_{obs}$ is the number of observed events. If the $CL_{s+b}$ is below 5\% we exclude that we can find any new physics with the selection criteria we have applied. Putting equation \ref{eq:signifobs} and \ref{eq:CL} together we get the expression

\begin{equation}
    \label{eq:signif}
    Z_N = \Phi^{-1}(1-CL_{s+b}),
\end{equation}

where $Z_N$ is the expected significance. To claim exclusion in the signal region we need a $CL_{s+b} \leq$ 5\% which corresponds to $Z_N \leq 1.64 \sigma$. In this thesis the $Z_N$ is calculated by an already existing tool in ROOT called \texttt{RooStats.NumberCountingUtils.BinomialExpZ} which simply takes the number of events for both background and signal as input together with a systematic background uncertainty of e.g. 20\%.







\begin{comment}

\begin{equation}
    \label{eq:pval3}
    p = P(n \geq q_{obs}|b) \sum_{n=q_{obs}}^\infty \frac{b^n}{n!} e^{-b}
\end{equation}

As mentioned several times in this project the goal is to claim discovery of the Higgs boson. To do that we have to reject the b-only hypothesis. To do this we are going to use p-values and confidence levels. \\


which is a probability of finding $n$ events in our signal region. 









\textbf{P-values}\\
Assuming a null hypothesis to be true we can use the p-value as a function that quantifies how often we would obtain data as far away from the null hypothesis as the observed data. Since it is a function of the data, the p-value is of course a random variable. We have the p-values $p_0$ and $p_1$ for the hypotheses $H_0$ and $H_1$ which is given by 

\begin{equation}
    \label{eq:p0}
    p_0 = \int_{t_{obs}}^{+ \infty} g(t|H_0) dt,
\end{equation}
and
\begin{equation}
    \label{eq:p1}
    p_1 = \int_{t_{obs}}^{+ \infty} g(t|H_1) dt.
\end{equation}

These are probabilities to find the test statistic ($t$) values that are equal or greater than the observed test statistic ($t_{obs}$). The p-values can also be expressed by confidence levels as

\begin{equation}
    \label{eq:p0CL}
    p_0 = 1-CL_b
\end{equation}
and 
\begin{equation}
    \label{eq:p1CL}
    p_1 = CL_{s+b}.
\end{equation}

How to calculate the $CL_b$ and $CL_{s+b}$ is shown in equation \ref{eq:CLb} and \ref{eq:CLsb} later in this section.\\

\textbf{Confidence levels}\\
The method we are using to check if this is a good fit is confidence levels. The confidence level tells us how often the results will fit the expectations. E.g. a confidence level of 90\% tells us that the results fit the expectations 90\% of the time. In figure \ref{fig:CL} we can see an illustrative plot of the test statistic distribution with the confidence levels marked. 


\begin{figure}[H]
    \centering
    \includegraphics[width = 0.7\textwidth]{Figures/CL.png}
    \caption{This is a illustration of the test statistic distribution for the b-only and s+b hypotheses with the $CL_b$ and $CL_{s+b}$ marked. \cite{datanal}}
    \label{fig:CL}
\end{figure}

The compability of the observed test statistic, $t_{obs}$, with the two hypothesis, b-only and s+b, are defined as

\begin{equation}
    \label{eq:CLb}
    1 - CL_b = \int_{-\infty}^{t_{obs}} \text{g(t; b-only) dt}
\end{equation}

and

\begin{equation}
    \label{eq:CLsb}
    CL_{s+b} = \int_{t_{obs}}^{+\infty} \text{g(t; s+b) dt}.
\end{equation}



\textbf{Results for the p-values and confidence levels}\\
By using equation \ref{eq:CLb} and \ref{eq:CLsb} we get the following results



In table \ref{tab:CLZ} we can see the results for the p-values ($1-CL_b$) and the confidence levels. As we can see is the expected significance is below 5$\sigma$ so we did not expect to claim a discovery which also is compatible with the observed significance from the data. \\

We can also see that the value for $CL_{s+b}$ for the b-only hypothesis is below 0.05 which is the limit for excluding this hypothesis but since the observed value is $> 0.05$ we can't exclude it either.  



\begin{table}[H]
    \centering
    \begin{tabular}{c c c}\hline
     & 1-CL_b & CL_{s+b}\\\hline\hline
    b-only     &  & \\
    s+b     &  &\\
    data &  & \\\hline
    \end{tabular}
    \caption{Caption}
    \label{tab:CL}
\end{table}


\begin{table}[H]
    \centering
    \begin{tabular}{c c}\hline
     & CL$_{s+b}$ \\\hline\hline
    Median s+b experiment     &  0.500\\
    Median b-only experiment     & 0.023\\
    Data & 0.843\\\hline
    \end{tabular}
    \caption{Caption}
    \label{tab:my_label}
\end{table}


\end{comment}



\section{Results}
In this section the expected significance for all four new physics processes and the three analysis methods are presented. The significance is calculated as described in the previous section. For the ML methods we have studied the three benchmark signals (from high, intermediate and low mass splittings) for each process and optimized the sensitivity by applying a cut at each bin in figure \ref{fig:BDTdataAll} and \ref{fig:NNdataAll}. The optimization procedure starts by calculating the significance when integrating over all the bins (i.e complete distribution of ML scores from 0 to 1). Then we move to the next bin, integrate up to 1 and compute the corresponding significance. This procedure is repeated until only the last bin remains in the significance calculation. At the end we see which cut in the ML score gave the best significance and use this to calculate the significance for all signals for the respective mass splitting. The bin which is used to place the cut for each optimization varies a bit, but is usually within the last 10\% of the bins in figure \ref{fig:BDTdataAll} and \ref{fig:NNdataAll}. This is expected since we see that we have a lot more signal in this region of the output than in the first bins. 

\subsection{Direct slepton production}
\label{sec:resSlepSlep}
The first results we are going to look at is for the direct slepton production. The results are presented in figure \ref{fig:signAllSlepSlep}, where we have the mass of the slepton on the x-axis and the mass of the neutralino on the y-axis. 

\begin{figure}[H]
    \centering
    \begin{subfigure}[t!]{0.49\textwidth}
    \includegraphics[width = \textwidth]{Figures/Significances/significanceCutandCount_slepslep_all.pdf}
    \caption{Cut and count.}
        \label{fig:signAllSlepSlepcandc}
    \end{subfigure}
    \\
    \begin{subfigure}[t!]{0.49\textwidth}
    \includegraphics[width = \textwidth]{Figures/Significances/significance_BDT_slepslep_All_level.pdf}
    \caption{Boosted Decision Tree.}
        \label{fig:signAllSlepSlepBDT}
    \end{subfigure}      
    \begin{subfigure}[t!]{0.49\textwidth}
    \includegraphics[width = \textwidth]{Figures/Significances/significance_NN_slepslep_All_level.pdf}
    \caption{Neural Network.}
        \label{fig:signAllSlepSlepNN}
    \end{subfigure}
    \caption{Significance plots for direct slepton production where all features are used during training the ML models. Since we have an overlap on some of the samples in the lower left corner, we have zoomed in at this area and added it in a empty area in the upper left corner.}
    \label{fig:signAllSlepSlep}
\end{figure}

This figure shows the expected significance for cut and count (\ref{fig:signAllSlepSlepcandc}), BDT (\ref{fig:signAllSlepSlepBDT}) and NN (\ref{fig:signAllSlepSlepNN}). The color bar is fixed at 3.5 in all plots to make it easier to compare the different results. Exactly how much greater than 3.5 the significance is is not so interesting since for these points we already have very good sensitivity (the 95\% CL exclusion limit, in figure \ref{fig:exclusionPlots}, corresponds to a significance of 1.63). Nevertheless, the exact significance value for each point is shown in the plot for each point. As we can see, the expected significance is greater for several signal samples for both the BDT and NN compared with the cut and count method. In particular, signal samples with low mass splittings ($\Delta m \leq 100$ GeV), found along the diagonal in figure \ref{fig:signAllSlepSlep}, show big improvements for both ML methods compared to cut and count. As stated in section \ref{sec:summary_ML} signals with low mass splittings are particularly hard to distinguish from SM background, making these results very interesting.












\subsection{Chargino pair with slepton/sneutrino-mediated-decay}
\label{sec:resC1C1_SlepSnu}

The results for the chargino production with slepton/sneutrino-mediated-decays are presented in figure \ref{fig:signAllslepsnu}. Here are the x-axis the mass of the chargino, while the y-axis remains the mass of the neutralino.

\begin{figure}[H]
    \centering
    \begin{subfigure}[t!]{0.49\textwidth}
    \includegraphics[width = \textwidth]{Figures/Significances/significanceCutandCount_slepsnu_all.pdf}
    \caption{Cut and count.}
        \label{fig:signAllslepsnucandc}
    \end{subfigure}
    \\
    \begin{subfigure}[t!]{0.49\textwidth}
    \includegraphics[width = \textwidth]{Figures/Significances/significance_BDT_slepsnu_All_level.pdf}
    \caption{Boosted Decision Tree.}
        \label{fig:signAllslepsnuBDT}
    \end{subfigure}      
    \begin{subfigure}[t!]{0.49\textwidth}
    \includegraphics[width = \textwidth]{Figures/Significances/significance_NN_slepsnu_All_level.pdf}
    \caption{Neural Network.}
        \label{fig:signAllslepsnuNN}
    \end{subfigure}
    \caption{Significance plots for chargino production with slepton/sneutrino-mediated-decay where all features are used during training of the ML models.}
    \label{fig:signAllslepsnu}
\end{figure}

Figure \ref{fig:signAllslepsnu} shows the expected significance for all three analysis methods used in this thesis. As for the direct slepton production, we can see that the ML have a greater sensitivity than the cut and count. The ML methods also have a greater sensitivity for low mass splittings ($\Delta m < 200$ GeV) compared to cut and count. 





























\subsection{Chargino pair with W-boson-mediated-decay}
\label{sec:resC1C1_WW}

The next results we are going to look at are for the chargino production with W-boson-mediated-decay, which are presented in figure \ref{fig:signAllWW}. The x- and y-axis are the same as for the chargino production with slepton/sneutrino-mediated-decay. 

\begin{figure}[H]
    \centering
    \begin{subfigure}[t!]{0.49\textwidth}
    \includegraphics[width = \textwidth]{Figures/Significances/significanceCutandCount_WW_all.pdf}
    \caption{Cut and count.}
        \label{fig:signAllWWcandc}
    \end{subfigure}
    \\
    \begin{subfigure}[t!]{0.49\textwidth}
    \includegraphics[width = \textwidth]{Figures/Significances/significance_BDT_WW_All_level.pdf}
    \caption{Boosted Decision Tree.}
        \label{fig:signAllWWBDT}
    \end{subfigure}      
    \begin{subfigure}[t!]{0.49\textwidth}
    \includegraphics[width = \textwidth]{Figures/Significances/significance_NN_WW_All_level.pdf}
    \caption{Neural Network.}
        \label{fig:signAllWWNN}
    \end{subfigure}
    \caption{Significance plots for chargino production with W-boson-mediated-decay where all features are used during training the ML models.}
    \label{fig:signAllWW}
\end{figure}

From figure \ref{fig:signAllWW}, we can easily see that the sensitivity are better for most signal samples for the ML methods compared with the cut and count method. However, the sensitivity over all is not very good, and we should not expect to make any discoveries at any mass splitting. This is somewhat expected from the fact that the cross-section for W-mediated decay is much smaller since we loose quite a lot by ignoring the quark decays of the W (only 11\% of the BR is to leptons). In addition the W-mediated decay is identical to SM WW production which is a challenging background. More optimization would be needed to increase the sensitivity in this channel, which also are discussed in the publication \cite{sleptonexclusion}.


































\subsection{Mono-Z}
\label{sec:resMono-Z}

The last results we are going to look at is for the mono-Z process. The results are presented in figure \ref{fig:signAllmonoZ}, where we have the mass of the mediator on the x-axis and the mass of the DM particle on the y-axis. 

\begin{figure}[H]
    \centering
    \begin{subfigure}[t!]{0.49\textwidth}
    \includegraphics[width = \textwidth]{Figures/Significances/significanceCutandCount_monoZ_all.pdf}
    \caption{Cut and count.}
        \label{fig:signAllmonoZcandc}
    \end{subfigure}
    \\
    \begin{subfigure}[t!]{0.49\textwidth}
    \includegraphics[width = \textwidth]{Figures/Significances/significance_BDT_monoZ_All_level.pdf}
    \caption{Boosted Decision Tree.}
        \label{fig:signAllmonoZBDT}
    \end{subfigure}      
    \begin{subfigure}[t!]{0.49\textwidth}
    \includegraphics[width = \textwidth]{Figures/Significances/significance_NN_monoZ_All_level.pdf}
    \caption{Neural Network.}
        \label{fig:signAllmonoZNN}
    \end{subfigure}
    \caption{Significance plots for the mono-Z process where all features are used during training the ML models.}
    \label{fig:signAllmonoZ}
\end{figure}

The figure shows that the cut and count method is somewhat sensitive to signals with low mass splittings ($\Delta m < 100$ GeV). However, the ML methods are still more sensitive to a wider range of signals with low mass splittings. 



\subsection{Summarizing the results}
From our studies it seems that overall the ML methods show a better sensitivity compared to the cut and count method, especially for low mass splittings. Low mass splittings is what we are most interested in improving with ML techniques since the cut and count analyses are less sensitive here. This is also the case for the ML methods trained on low level and high level features, but they are not as good as the ones trained on all features. These results can be found in appendix \ref{sec:appsignificance}. 

The difference between the performance of the BDT and NN are not very noticeable, but it seems like the BDT is slightly more sensitive overall. But, as mentioned earlier in this thesis, the NN are much more sensitive to the parameters we give it than the BDT. This means that with some more time for optimization of the hyperparameters, we would probably be able to get just as good results as for the BDT or even better. 

In this thesis we have not done a full analysis of the cut and count method as is presented in chapter \ref{sec:CandCanalysis}. We have only used one signal region (for SUSY $m_{T2} > 160$ GeV) while in they publication they do a multi-bin likelihood fit including the control and signal regions. This is why the exclusion curves from the publications are better than what we have obtained for the cut and count. Nevertheless, the fact that the ML methods show a better performance towards low mass splitting is very promising and with more optimization the sensitivity could be further increased. For direct slepton, especially, it seems like we are able to get a sensitivity which is better than what was achieved in the publication in the lower left part of the parameter space.

\begin{comment}

- Dir slep: I LL  har ML litt dårlig sensitivitet en AL, HL har enda dårligere - forventet pga dir slep. Høy level variabler med her er ikke relevante for prosess. 

-Slep/snu: I LL  har ML hakket dårlig sensitivitet en AL, HL er ok, ikke så stor forskjell på LL og HL. 

- WW: LL < HL < AL. Interessant. 

- Mono-Z: Ganske bra LL og HL, nesten like bra som AL.



\end{comment}





