\addcontentsline{toc}{chapter}{Introduction}
\chapter*{Introduction}
\label{sec:Introduction}
The Standard Model (SM) \cite{thomson} is a well known theory of particle physics lately further confirmed after the Higgs boson was discovered at CERN in 2012 by the ATLAS \cite{Higgs_ATLAS} and CMS \cite{Higgs_CMS} collaborations. The SM can describe all visible matter and contains matter particles, quarks and leptons, and the fundamental forces acting between them. Even though the SM can describe all the matter we can see around us, it can't describe the whole universe. There are of course many theories which are trying to describe the rest, but in this thesis we are mainly going to focus on Supersymmetry (SUSY). SUSY is an extension of the SM which predicts that every particle in the SM has a supersymmetric partner with equal quantum numbers, except for the spin. You will get a deeper explanation of this later. One of the consequences of discovering SUSY may be that it also can provide a possible explanation to Dark Matter (DM), which we are going to look more into in this thesis.

For many years the standard way to do an analysis have been to simply place cuts using a set of variables which efficiently can discriminate signal from background. Using a set of cuts one defines signal regions, optimized to increase the signal over background ratio. However, these methods have limitations when it comes to discovering correlations between variables. Cut and count methods usually apply cuts evaluating one variable at the time or uses so called rectangular cuts evaluating the correlation between maximum two variables. With computers and new algorithms constantly being improved and developed, particle physicists have tried to look for other possibilities to perform searches for new physics, namely by using Machine Learning (ML) methods. ML methods, in contrast to the cut and count methods, can investigate multidimensional correlations and place cuts using boundaries in higher dimensions. ML has therefore become very popular and is widely used in many fields of research. In this thesis we are going to look at a Boosted Decision Tree (BDT) and a Neural Network (NN), and compare the results against each other and with a more traditional cut and count analysis. 

In chapter \ref{sec:Theory} a short introduction to the SM and new physics will be given. Chapter \ref{sec:LHCandATLAS} contains a short presentation of the detector and computers used for this thesis. We will then move to the search for new physics with the cut and count method in chapter \ref{sec:CandCanalysis}. The next chapters (\ref{sec:ML} and \ref{sec:MLanalysis}) we will get to know ML and look at how the analysis was done. In the end we have presented our results in chapter \ref{sec:results} and conclusion in chapter \ref{sec:Conclusion}.

\begin{comment}
- CERN, LHC, ATLAS, kinematics, Computing thing
- Search for SUSY trad. + pub.
- ML intro
- ML anal
- Results
- Conclusion
To begin with a short introduction to the SM will be given, then an introduction of a new symmetry (Supersymmetry) and by considering an extension allowing DM particles in chapter \ref{sec:Theory}.


We move on to chapter \ref{sec:LHSandATLAS} where the experiment that have provided the data for this thesis will be presented. 


\begin{itemize}
    \item {\bf SOME preliminary comments from Farid }
    \item Good thing: you have more or less touched at most of the things + the analysis / results are converging.
    \item Theory: Without going into great details, try to be exact in the formulation of the SM and clearly say why we search for physics beyond the SM. The SM is based on the symmetry groups $U(1)_Y * SU(2)_L * SU(3)_C$. Talk about the important quantum numbers, Isospin, according to which the structure (up, down) is justified. Present the leptons and quarks (quarks are mixed, CKM matrix, ..., neutrinos are mixed, PMNS matrix) and give the important properties according to the interactions whose properties / role should be described. Talk about the gauge bosons, massless photon and gluons and massive $W^±$ and $Z^o$, talk about the breaking of the $U(1)_Y * SU(2)_L $ symmetry through the introduction of a scalar Higgs doublet field, ...  Very basic, short but concise :).   
    \item Tell about the analyses you want to reproduce and confront to BDT and NN.Introduce the motivation and quick selection and idea behind the analyses, reproduce the table with selection cuts, and description. Reproduce the MET and/or MT2 distributions obtained by ATLAS, ... 
    \item Structure well your chapters and sections, with introduction to (and summary of) the chapters, ...
    \item In general describe plots and tables and discuss the results obtained: (very) good (enough) agreement, reason for some discrepancies, and consequences, ... 
    \item Introduction and chapter(s) on machine learning: I suggest to have a standalone chapter on Introduction to Machine Learning just after the chapter on The SM and beyond.This can be followed by a chapter "Search for Supersymmetry and DM in events with dileptons and missing energy: short motivation and description of both analysis (published results by ATLAS). Here you define the variables, cuts, show results (Mll and/or MT2 and/or MET, ...) from ATLAS. Optionally you can add in this chapter your own cut and count results. In the next chapter you describe the BDT and NN analyses. A final chapter between (Conclusion and outlook) will a comparison between the c&c and NNT/BDT analyses/results. 
    \item THAT were some suggestions ... We can discuss them tomorrow if needed
    \item more will come here ... 
\end{itemize}





\begin{itemize}
    \item Kort introduksjon om selve oppgaven
    \item Beskrivelse hvordan oppgaven er satt opp
    \item Forklar hvorfor teori er kort pga at ML er nytt
\end{itemize}


\end{comment}
