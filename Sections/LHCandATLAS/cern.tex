\section{CERN}
\label{sec:cern}

The European organisation for nuclear research, CERN, started as a research facility for mainly nuclear physics. It was built on the border between France and Switzerland near Geneva in 1954. CERN has 22 member states, where Norway is one of the founding members, but it welcomes people from all over the world to take part in the different experiments and accelerator developments. 

CERN quickly became the biggest and leading research centre in particle physics as well, and the most famous discoveries done at CERN are from high energy particle collisions. Some of the biggest discoveries at CERN are: the weak neutral currents mediated by the hypothetical Z-boson in 1973 \cite{Zmed} and of course the discovery of the actual Z- and the W$^\pm$-bosons, mediators of the weak force in 1983/84 \cite{WZ1, WZ2, WZ3}. The most famous and recent discovery is certainly that of a first scalar boson, the Higgs boson in 2012 by the ATLAS and CMS experiments\cite{Higgs_ATLAS, Higgs_CMS}. The Higgs boson was the missing piece to confirm the Standard Model of particle physics.

Throughout the years many different accelerators have been built at CERN. E.g. the Super Proton Synchrotron (SPS), which are still in use, accelerates and collides protons and antiprotons, and enabled the UA1 and UA2 experiments to discover the Z- and W-bosons. The Large electron positron collider (LEP) was built in a 27 km long tunnel about 100 m below ground and was the largest accelerator at the time. LEP allowed important SM precision measurements, in particular confirming the presence of exactly three low mass neutrino flavours in the SM and stringent limits on the top and Higgs masses. The last run at LEP was done in 2000 paving the road for the start of the building of the Large Hadron Collider (LHC). LHC is the accelerator producing the proton-proton collisions being recorded by the ATLAS detector and which have been used in the analysis presented in this thesis. 