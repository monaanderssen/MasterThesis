\section{CERN}
\label{sec:cern}

The European organisation for nuclear research, CERN, started as a research facility on mainly nuclear physics. It was built on the border between France and Switzerland near Geneva in 1954. CERN have 22 member states, where Norway is one of the founding members, but it welcomes people from all over the world to take part in the different experiments and accelerator development. 

CERN quickly became the biggest and leading research centre in particle physics as well, and the most famous discoveries done at CERN are the discoveries done with high energy particle collisions. To mention some of the discoveries done at CERN we have weak neutral currents that are mediated by the hypothetical Z-boson in 1973 \cite{Zmed} and of course the discovery of the actual Z-boson and the W-boson mediators of the weak force in 1983/84 \cite{WZ1, WZ2, WZ3}. The most recent discovery is certainly that of the Higgs boson \cite{Higgs_ATLAS, Higgs_CMS} in 2012 by the ATLAS and CMS experiments, which also was the last piece to confirm the SM.

Through out the years there have been built many different accelerators at CERN. Among others the Super Proton Synchrotron (SPS) accelerated protons and antiprotons and allowed the UA1 and UA2 experiments to discover the Z- and W-bosons. The Large electron positron collider (LEP) was built in a 27 km long tunnel 100 m below ground and was definitively the biggest accelerator at the time. LEP allowed important SM precision measurements. The last run at LEP was done in 2000 paving the road for the start of building the Large Hadron Collider (LHC) which is the accelerator that produced proton-proton collisions data recorded by the ATLAS detector and which have been used to obtain the data for this thesis. 