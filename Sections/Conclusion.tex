\chapter{Conclusion and outlook}
\label{sec:Conclusion}

In this thesis we have been searching for SUSY and DM using two ML algorithms, namely BDT and NN. We have also reproduced the results from the publications done by ATLAS \cite{sleptonexclusion, monoZexclusion}, which have searched for the same signal processes using a simple cut and count method. The results from the cut and count method are used to evaluate how well the ML methods performed by comparing the expected sensitivity. The BDT and NN were trained on different compositions of mass splittings and features, and for four different signal processes (three SUSY and one DM). Both ML methods have overall performed very well and reached AUC-scores above 0.90 for every trained model. 

Looking at the achieved sensitivity of the three methods and comparing them to each other it is clear that the ML methods overall have more sensitivity for the signals than the cut and count. In particular for low mass splittings. To achieve a high sensitivity for low mass splittings is difficult in the cut and count method and it is therefore very satisfying that the ML perform very well for these signals. We can, with these results, conclude that ML may indeed be efficient and rewarding technique when performing searches for new physics phenomena. 

For future research using these or similar ML methods, it would be interesting to see what we could done to make the performance better. In this thesis we have used a couple of precuts, which will make the ML methods somewhat biased and thus not able to learn everything from a less selective data input. It would be interesting to see how it would perform without these precuts and with more features included. However, this would imply that we need a lot more computing power and time to perform our analysis because of the massive datasets. It would also be interesting to do a hyper parameter scan to see which parameters are the best to use for the BDT and NN. 







\begin{comment}


Konklusjon:
- ML mer sensitiv
- Si kort om hva som er gjort


Outlook:
- No precuts
- Hyperparameter søk
- Anomaly detection
- Andre/flere features
- 


\end{comment}
