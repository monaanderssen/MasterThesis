\chapter{Conclusion and outlook}
\label{sec:Conclusion}

In this thesis we have been searching for SUSY and DM using two ML algorithms, namely BDT and NN. We have also reproduced the results from the publications done by ATLAS \cite{sleptonexclusion, monoZexclusion}, which have looked at the same processes as us with the cut and count method. This is done to see how well the ML have performed compared to cut and count. The BDT and NN was trained on different compositions of mass splittings and features, and for four different processes (three SUSY and one DM). Both ML methods have overall performed very well and gotten an AUC-score above 0.90 for every trained model. 

The final results of this thesis was looking at the sensitivity for the three methods used, and compare them to each other. This resulted in that the ML methods overall have more sensitivity for the signals than the cut and count, especially for low mass splittings. Sensitivity for low mass splittings have been difficult for the cut and count method earlier and it is therefore very satisfying that the ML perform very well for these signals. We can, with these results, conclude that ML can be very efficient and rewarding to implement when doing a search for new physics phenomena. 

For future research with using these ML methods, it would be interesting to see what we could do to make the performance better. In this thesis we have used a couple of precuts, which will make the ML methods somewhat biased and not learn everything from the raw data input. It would be interesting to see how it would perform without these and some more features, but it also implies that we need a lot of computing power an time to do it because of the massive datasets. It would also be interesting to do a hyper parameter search\improvement{Tror ikke det heter det....} to see what parameters are the best to send into the BDT and NN. 







\begin{comment}


Konklusjon:
- ML mer sensitiv
- Si kort om hva som er gjort


Outlook:
- No precuts
- Hyperparameter søk
- Anomaly detection
- Andre/flere features
- 


\end{comment}
